\documentclass[12pt]{article}
\usepackage[T1]{fontenc}
\usepackage[utf8]{inputenc}
\usepackage{polski}
\usepackage{minted}
\usepackage{geometry}
\usepackage{natbib}
\usepackage{enumitem}
\usepackage{graphicx}
\usepackage{bold-extra}
\usepackage[font=small,labelfont=bf]{caption}
\usepackage{hyperref}
\usepackage{titlesec}
\usepackage{indentfirst}
\usepackage{diagbox}
\usepackage{pifont}
\usepackage{amssymb}
\newcommand{\cmark}{\ding{51}}%
\newcommand{\xmark}{\ding{55}}%

\hyphenpenalty=10000
\tolerance=1000 \emergencystretch=2em
\titlelabel{\thetitle.\quad}



\setminted[text]{
                fontsize=\scriptsize
                }

 \geometry{
     left=23mm,
     top=25mm,
     right=23mm
 }


\def\mydate{\leavevmode\hbox{\twodigits\day.\twodigits\month.\the\year}}
\def\twodigits#1{\ifnum#1<10 0\fi\the#1}

\begin{document}
%titlepage
\thispagestyle{empty}
\begin{center}
\begin{minipage}{0.75\linewidth}
    \centering
    \includegraphics[width=0.45\linewidth]{agh_logo2.png}
    \par
    \vspace{2cm}
    {\bfseries{\scshape{\Huge  Systemy rozproszone}}}
    \par
    \vspace{1.7cm}
    {\scshape{\Large Laboratorium 2}}
    \par
    \vspace{0.8cm}
    {\scshape{\Large RabbitMQ}}
    \par
    \vspace{3cm}

    {\scshape{\Large Albert Gierlach}}\par
    \vspace{0.6cm}
    {\scshape{\Large nr. indeksu: 305303}}\par
    \vspace{1cm}

    {\Large \mydate}
\end{minipage}
\end{center}
\clearpage



\section{Zadanie 1a}
Stosowanie potwierdzania po otrzymaniu wiadomości powoduje sytuację, w której konsument może nie przetworzyć wiadomości, a zostanie ona potwierdzona, więc de facto wiadomość przepadła. Natomiast potwierdzanie po przetworzeniu daje nam dużą pewność, że konsument przetworzył naszą wiadomość. W przypadku, gdy proces konsumenta zostanie zrestartowany w trakcie przetwarzania, to po ponownym połączeniu, wiadomość zostanie ponownie do niego dostarczona, ponieważ producent nie otrzymał potwierdzenia przetworzenia wiadomości.

W związku z powyższym - większą niezawodność daje nam zastosowanie potwierdzeń po przetworzeniu.

Jeżeli nie będziemy potwierdzać wiadomości ani po otrzymaniu, ani po przetworzeniu, to każdy nowy proces konsumenta, będzie ponownie otrzymywał te same wiadomości, ponieważ nie zostaną one usunięte z kolejki wiadomości serwera.


\section{Zadanie 1b}
QoS wyłączone: \\

\begin{tabular}{| p{7cm} | p{7cm} |}
\hline
Konsument nr. 1 & Konsument nr. 2 \\
\hline
\begin{minted}{text}
Z1 CONSUMER
Waiting for messages...
Received: 1
Received: 1
Received: 1
Received: 1
Received: 1
\end{minted}
& \begin{minted}{text}
Z1 CONSUMER
Waiting for messages...
Received: 5
Received: 5
Received: 5
Received: 5
Received: 5
\end{minted} 
\\
\hline
\end{tabular}


\vspace{0.7cm}
QoS włączone: \\

\begin{tabular}{| p{7cm} | p{7cm} |}
\hline
Konsument nr. 1 & Konsument nr. 2 \\
\hline
\begin{minted}{text}
Z1 CONSUMER
Waiting for messages...
Received: 5
Received: 1
Received: 5
Received: 1
Received: 5
\end{minted}
& \begin{minted}{text}
Z1 CONSUMER
Waiting for messages...
Received: 1
Received: 1
Received: 5
Received: 1
Received: 5
\end{minted} 
\\
\hline
\end{tabular}
\vspace{1cm}

\section{Zadanie 2}

W celu przetestowania różnych polityk routingu stworzone zostały trzy kolejki z różnymi kluczami. Producent będzie wysyłał wiadomości z różnymi kluczami routingu, aby zweryfikować róznice. W tabelach zaznaczono także, która kolejka odbierze daną wiadomość.

\vspace{0.5cm}
Direct routing:

\vspace{0.3cm}
\begin{tabular}{|c|>{\centering\arraybackslash}p{2cm}|>{\centering\arraybackslash}p{2cm}|>{\centering\arraybackslash}p{2cm}|}
\hline
\backslashbox{klucz wiadomości}{klucz kolejki} & klucz.a & klucz.* & klucz.\# \\ \hline
klucz.a    & \cmark & \xmark & \xmark \\ \hline
klucz.b    & \xmark & \xmark & \xmark \\ \hline
klucz.c.d  & \xmark & \xmark & \xmark \\ \hline
klucz.\#   & \xmark & \xmark & \cmark \\ \hline
\end{tabular}\\\\

Powyższa tabela ukazuje naturę direct routingu - klucz wiadomości musi dokładnie pasować do klucza kolejki, a znaki specjalne typu *, \# nie mają zastosowania w tym typie routingu i są traktowane jak zwykłe znaki.

\vspace{1.5cm}
Topic routing:

\vspace{0.3cm}
\begin{tabular}{|c|>{\centering\arraybackslash}p{2cm}|>{\centering\arraybackslash}p{2cm}|>{\centering\arraybackslash}p{2cm}|}
\hline
\backslashbox{klucz wiadomości}{klucz kolejki} & klucz.a & klucz.* & klucz.\# \\ \hline
klucz.a    & \cmark & \cmark & \cmark \\ \hline
klucz.b    & \xmark & \cmark & \cmark \\ \hline
klucz.c.d  & \xmark & \xmark & \cmark \\ \hline
klucz.\#   & \xmark & \cmark & \cmark \\ \hline
\end{tabular}\\\\

W przypadku routingu typu Topic sytuacja wygląda całkiem inaczej. Widzimy zastosowanie wzorców - operator * dopasowuje dokładnie jedno dowolne słowo, a operator \# dopasowuje wiele (bądź zero) słów. W przypadku klucza wiadomości: \emph{klucz.c.d} kolejka przyjmująca następujące klucze: \emph{klucz.*} nie odebrała wiadomości, ponieważ zawierała ona dwa dodatkowe słowa, zamiast jednego.


\end{document}
